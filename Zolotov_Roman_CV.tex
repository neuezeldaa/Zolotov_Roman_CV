\documentclass[a4paper,10pt]{article}
\usepackage[margin=1in]{geometry}
\usepackage{enumitem}
\usepackage{hyperref}
\usepackage{titlesec}
\usepackage{fontawesome5}
\usepackage[T2A, T1]{fontenc}
\usepackage[utf8]{inputenc}
\usepackage[russian, english]{babel}
\usepackage{fontspec} 

\defaultfontfeatures{Ligatures={TeX},Renderer=Basic} 
\setmainfont[Ligatures={TeX,Historic}]{Times New Roman}
\selectlanguage{russian}

% Formatting
\titleformat{\section}{\large\bfseries}{}{0em}{}[\titlerule]
\setlist[itemize]{noitemsep, topsep=0pt}

\begin{document}

%-------------------- Заголовок --------------------
\begin{center}
    {\LARGE \textbf{Золотов Роман Владимирович}}\\
    \vspace{2mm}
    \normalsize \textbf {ML | Аналитик}\\
    \vspace{2mm}
    \faMapMarker*\ Санкт-Петербург\quad
    \faEnvelope\ \href{mailto:z_roman_05@mail.ru}{z\_roman\_05@mail.ru} \quad
    \faPhone\ +7 (962) 630-93-76 \quad\\
    \vspace{2mm}
    \faTelegramPlane\ \href{https://t.me/neuezeldaa}{https://t.me/neuezeldaa} \quad
    \faGithub\ \href{https://github.com/neuezeldaa}{github.com/neuezeldaa} \\
\end{center}

%-------------------- Профиль --------------------
\section*{\faMale \hspace{0.5em} Профиль}
Студент 3 курса СПбГЭТУ "ЛЭТИ" по специальности "Радиотехника. Системы компьютерного зрения". Имею опыт в Машинном Обучении, визуализации и анализе данных, работы с фильтрами и контурами в компьютерном зрении.
Проводил полный цикл A/B тестирования. Писал программы на Python и C++. Готов расширять знания в теории и применять их на практике.

%-------------------- Навыки --------------------
\section*{\faWrench \hspace{0.5em} Технические навыки}
\begin{tabular}{ l l }
    \textbf{Языки Программироавния:} & Python, C++, Go, SQL \\
    \textbf{Фреймворки:} & PyTorch, OpenCV, MFC, scikit-learn, mathplotlib, numpy \\
    \textbf{Инструменты:} & Git,, YouGile, Linux, Jupyter Notebook (Google colab), LaTeX\\
\end{tabular}

%-------------------- Образование --------------------

\section*{\faUniversity \hspace{0.5em} Образование}
\textbf{СПбГЭТУ ЛЭТИ, Санкт-Петербург} \hfill \textit{ 2023-2027} \\
\textit{Бакалавриат "Радиотехника. Системы компьютерного зрения"} \\

%-------------------- Проекты --------------------
\section*{\faProjectDiagram \hspace{0.5em} Pet-проекты} 
\begin{itemize}
    \item \textbf{Статья с решением задачи Titanic на Kaggle для новичков в ML} \hfill 2025\\
    \textit{Python, Jupyther Notebook, PyTorch, Kaggle}
    \begin{itemize}
        \item[--] Пошаговый анализ признаков с визуализацией.
        \item[--] Реализация и обучение Многослойной Нейронной сети.
        \item[--] Применение методов машинного обучения (LR, RandomForest и др.).
    \end{itemize}

    \item \textbf{Отслеживание перемещения объекта} \hfill 2025\\
    \textit{C++, Python, OpenCV, Linux}
    \begin{itemize}
        \item[--] Использование фильтров для выявления контура объекта.
        \item[--] Отсчет времени перемещения объекта между двумя точками.
        \item[--] Отслеживание объекта идет с веб-камеры.
    \end{itemize}

    \item \textbf{Страница на GitHub с решением задач на CodeRun} \hfill 2025\\
    \textit{C++, Python, Jupyter Notebook, CodeRun}
    \begin{itemize}
        \item[--] Удобное представление списка задач в Readme.md 
        \item[--] Ознакомление с алгоритмами и структурами данных
	 \item[--] Широкий спектр тематик задач (Аналитика, Backend, ML) 
    \end{itemize}

    \item \textbf{Разработка графического приложения для построения графиков функций} \hfill 2025\\
    \textit{C++, MFC, ООП}
    \begin{itemize}
        \item[--] Оконное графическое приложение с MFC
        \item[--] User-friendly интерфейс
    \end{itemize}
\end{itemize}

%-------------------- Дополнительно --------------------
\section*{Дополнительная информация}
\begin{itemize}
    \item Увлечения: Алгоритмы, Математика, Спорт 
\end{itemize}

\end{document}